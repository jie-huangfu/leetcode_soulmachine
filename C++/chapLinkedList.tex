\chapter{单链表}

\section{单链表} %%%%%%%%%%%%%%%%%%%%%%%%%%%%%%

单链表节点的定义如下:
\begin{Code}
// 单链表节点
struct ListNode {
    int val;
    ListNode *next;
    ListNode(int x) : val(x), next(nullptr) { }
};
\end{Code}


\subsection{Add Two Numbers}
\label{sec:add-two-numbers}


\subsubsection{描述}
You are given two linked lists representing two non-negative numbers. The digits are stored in reverse order and each of their nodes contain a single digit. Add the two numbers and return it as a linked list.

Input: {\small \fontspec{Latin Modern Mono} (2 -> 4 -> 3) + (5 -> 6 -> 4)}

Output: {\small \fontspec{Latin Modern Mono} 7 -> 0 -> 8}


\subsubsection{分析}
跟Add Binary(见 \S \ref{sec:add-binary})很类似


\subsubsection{代码}
\begin{Code}
// LeetCode, Add Two Numbers
// 跟Add Binary 很类似
// 时间复杂度O(m+n),空间复杂度O(1)
class Solution {
public:
    ListNode *addTwoNumbers(ListNode *l1, ListNode *l2) {
        ListNode dummy(-1); // 头节点
        int carry = 0;
        ListNode *prev = &dummy;
        for (ListNode *pa = l1, *pb = l2;
             pa != nullptr || pb != nullptr;
             pa = pa == nullptr ? nullptr : pa->next,
             pb = pb == nullptr ? nullptr : pb->next,
             prev = prev->next) {
            const int ai = pa == nullptr ? 0 : pa->val;
            const int bi = pb == nullptr ? 0 : pb->val;
            const int value = (ai + bi + carry) % 10;
            carry = (ai + bi + carry) / 10;
            prev->next = new ListNode(value); // 尾插法
        }
        if (carry > 0)
            prev->next = new ListNode(carry);
        return dummy.next;
    }
};
\end{Code}


\subsubsection{相关题目}

\begindot
\item Add Binary, 见 \S \ref{sec:add-binary}
\myenddot


\subsection{Reverse Linked List II}
\label{sec:reverse-linked-list-ii}


\subsubsection{描述}
Reverse a linked list from position $m$ to $n$. Do it in-place and in one-pass.

For example:
Given \code{1->2->3->4->5->nullptr}, $m$ = 2 and $n$ = 4,

return \code{1->4->3->2->5->nullptr}.

Note:
Given m, n satisfy the following condition:
$1 \leq m \leq  n \leq $ length of list.


\subsubsection{分析}
这题非常繁琐,有很多边界检查,15分钟内做到bug free很有难度!


\subsubsection{代码}
\begin{Code}
// LeetCode, Reverse Linked List II
// 迭代版,时间复杂度O(n),空间复杂度O(1)
class Solution {
public:
    ListNode *reverseBetween(ListNode *head, int m, int n) {
        ListNode dummy(-1);
        dummy.next = head;

        ListNode *prev = &dummy;
        for (int i = 0; i < m-1; ++i)
            prev = prev->next;
        ListNode* const head2 = prev;

        prev = head2->next;
        ListNode *cur = prev->next;
        for (int i = m; i < n; ++i) {
            prev->next = cur->next;
            cur->next = head2->next;
            head2->next = cur;  // 头插法
            cur = prev->next;
        }

        return dummy.next;
    }
};
\end{Code}


\subsubsection{相关题目}

\begindot
\item 无
\myenddot


\subsection{Partition List}
\label{sec:partition-list}


\subsubsection{描述}
Given a linked list and a value $x$, partition it such that all nodes less than $x$ come before nodes greater than or equal to $x$.

You should preserve the original relative order of the nodes in each of the two partitions.

For example,
Given \code{1->4->3->2->5->2} and \code{x = 3}, return \code{1->2->2->4->3->5}.


\subsubsection{分析}
无


\subsubsection{代码}
\begin{Code}
// LeetCode, Partition List
// 时间复杂度O(n),空间复杂度O(1)
class Solution {
public:
    ListNode* partition(ListNode* head, int x) {
        ListNode left_dummy(-1); // 头结点
        ListNode right_dummy(-1); // 头结点

        auto left_cur = &left_dummy;
        auto right_cur = &right_dummy;

        for (ListNode *cur = head; cur; cur = cur->next) {
            if (cur->val < x) {
                left_cur->next = cur;
                left_cur = cur;
            } else {
                right_cur->next = cur;
                right_cur = cur;
            }
        }

        left_cur->next = right_dummy.next;
        right_cur->next = nullptr;

        return left_dummy.next;
    }
};
\end{Code}


\subsubsection{相关题目}

\begindot
\item 无
\myenddot


\subsection{Remove Duplicates from Sorted List}
\label{sec:remove-duplicates-from-sorted-list}


\subsubsection{描述}
Given a sorted linked list, delete all duplicates such that each element appear only once.

For example,

Given \code{1->1->2}, return \code{1->2}.

Given \code{1->1->2->3->3}, return \code{1->2->3}.


\subsubsection{分析}
无


\subsubsection{递归版}
\begin{Code}
// LeetCode, Remove Duplicates from Sorted List
// 递归版,时间复杂度O(n),空间复杂度O(1)
class Solution {
public:
    ListNode *deleteDuplicates(ListNode *head) {
        if (!head) return head;
        ListNode dummy(head->val + 1); // 值只要跟head不同即可
        dummy.next = head;

        recur(&dummy, head);
        return dummy.next;
    }
private:
    static void recur(ListNode *prev, ListNode *cur) {
        if (cur == nullptr) return;

        if (prev->val == cur->val) { // 删除head
            prev->next = cur->next;
            delete cur;
            recur(prev, prev->next);
        } else {
            recur(prev->next, cur->next);
        }
    }
};
\end{Code}


\subsubsection{迭代版}
\begin{Code}
// LeetCode, Remove Duplicates from Sorted List
// 迭代版,时间复杂度O(n),空间复杂度O(1)
class Solution {
public:
    ListNode *deleteDuplicates(ListNode *head) {
        if (head == nullptr) return nullptr;

        for (ListNode *prev = head, *cur = head->next; cur; cur = prev->next) {
            if (prev->val == cur->val) {
                prev->next = cur->next;
                delete cur;
            } else {
                prev = cur;
            }
        }
        return head;
    }
};
\end{Code}


\subsubsection{相关题目}

\begindot
\item Remove Duplicates from Sorted List II,见 \S \ref{sec:remove-duplicates-from-sorted-list-ii}
\myenddot


\subsection{Remove Duplicates from Sorted List II}
\label{sec:remove-duplicates-from-sorted-list-ii}


\subsubsection{描述}
Given a sorted linked list, delete all nodes that have duplicate numbers, leaving only distinct numbers from the original list.

For example,

Given \code{1->2->3->3->4->4->5}, return \code{1->2->5}.

Given \code{1->1->1->2->3}, return \code{2->3}.


\subsubsection{分析}
无


\subsubsection{递归版}
\begin{Code}
// LeetCode, Remove Duplicates from Sorted List II
// 递归版,时间复杂度O(n),空间复杂度O(1)
class Solution {
public:
    ListNode *deleteDuplicates(ListNode *head) {
        if (!head || !head->next) return head;

        ListNode *p = head->next;
        if (head->val == p->val) {
            while (p && head->val == p->val) {
                ListNode *tmp = p;
                p = p->next;
                delete tmp;
            }
            delete head;
            return deleteDuplicates(p);
        } else {
            head->next = deleteDuplicates(head->next);
            return head;
        }
    }
};
\end{Code}


\subsubsection{迭代版}
\begin{Code}
// LeetCode, Remove Duplicates from Sorted List II
// 迭代版,时间复杂度O(n),空间复杂度O(1)
class Solution {
public:
    ListNode *deleteDuplicates(ListNode *head) {
        if (head == nullptr) return head;

        ListNode dummy(INT_MIN); // 头结点
        dummy.next = head;
        ListNode *prev = &dummy, *cur = head;
        while (cur != nullptr) {
            bool duplicated = false;
            while (cur->next != nullptr && cur->val == cur->next->val) {
                duplicated = true;
                ListNode *temp = cur;
                cur = cur->next;
                delete temp;
            }
            if (duplicated) { // 删除重复的最后一个元素
                ListNode *temp = cur;
                cur = cur->next;
                delete temp;
                continue;
            }
            prev->next = cur;
            prev = prev->next;
            cur = cur->next;
        }
        prev->next = cur;
        return dummy.next;
    }
};
\end{Code}


\subsubsection{相关题目}

\begindot
\item Remove Duplicates from Sorted List,见 \S \ref{sec:remove-duplicates-from-sorted-list}
\myenddot


\subsection{Rotate List}
\label{sec:rotate-list}


\subsubsection{描述}
Given a list, rotate the list to the right by $k$ places, where $k$ is non-negative.

For example:
Given \code{1->2->3->4->5->nullptr} and \code{k = 2}, return \code{4->5->1->2->3->nullptr}.


\subsubsection{分析}
先遍历一遍,得出链表长度$len$,注意$k$可能大于$len$,因此令$k \%= len$。将尾节点next指针指向首节点,形成一个环,接着往后跑$len-k$步,从这里断开,就是要求的结果了。


\subsubsection{代码}
\begin{Code}
// LeetCode, Remove Rotate List
// 时间复杂度O(n),空间复杂度O(1)
class Solution {
public:
    ListNode *rotateRight(ListNode *head, int k) {
        if (head == nullptr || k == 0) return head;

        int len = 1;
        ListNode* p = head;
        while (p->next) { // 求长度
            len++;
            p = p->next;
        }
        k = len - k % len;

        p->next = head; // 首尾相连
        for(int step = 0; step < k; step++) {
            p = p->next;  //接着往后跑
        }
        head = p->next; // 新的首节点
        p->next = nullptr; // 断开环
        return head;
    }
};
\end{Code}


\subsubsection{相关题目}

\begindot
\item 无
\myenddot


\subsection{Remove Nth Node From End of List}
\label{sec:remove-nth-node-from-end-of-list}


\subsubsection{描述}
Given a linked list, remove the $n^{th}$ node from the end of list and return its head.

For example, Given linked list: \code{1->2->3->4->5}, and $n$ = 2.

After removing the second node from the end, the linked list becomes \code{1->2->3->5}.

Note:
\begindot
\item Given $n$ will always be valid.
\item Try to do this in one pass.
\myenddot


\subsubsection{分析}
设两个指针$p,q$,让$q$先走$n$步,然后$p$和$q$一起走,直到$q$走到尾节点,删除\fn{p->next}即可。


\subsubsection{代码}
\begin{Code}
// LeetCode, Remove Nth Node From End of List
// 时间复杂度O(n),空间复杂度O(1)
class Solution {
public:
    ListNode *removeNthFromEnd(ListNode *head, int n) {
        ListNode dummy{-1, head};
        ListNode *p = &dummy, *q = &dummy;

        for (int i = 0; i < n; i++)  // q先走n步
            q = q->next;

        while(q->next) { // 一起走
            p = p->next;
            q = q->next;
        }
        ListNode *tmp = p->next;
        p->next = p->next->next;
        delete tmp;
        return dummy.next;
    }
};
\end{Code}


\subsubsection{相关题目}

\begindot
\item 无
\myenddot


\subsection{Swap Nodes in Pairs}
\label{sec:swap-nodes-in-pairs}


\subsubsection{描述}
Given a linked list, swap every two adjacent nodes and return its head.

For example,
Given \code{1->2->3->4}, you should return the list as \code{2->1->4->3}.

Your algorithm should use only constant space. You may \emph{not} modify the values in the list, only nodes itself can be changed.


\subsubsection{分析}
无


\subsubsection{代码}
\begin{Code}
// LeetCode, Swap Nodes in Pairs
// 时间复杂度O(n),空间复杂度O(1)
class Solution {
public:
    ListNode *swapPairs(ListNode *head) {
        if (head == nullptr || head->next == nullptr) return head;
        ListNode dummy(-1);
        dummy.next = head;

        for(ListNode *prev = &dummy, *cur = prev->next, *next = cur->next;
                next;
                prev = cur, cur = cur->next, next = cur ? cur->next: nullptr) {
            prev->next = next;
            cur->next = next->next;
            next->next = cur;
        }
        return dummy.next;
    }
};
\end{Code}

下面这种写法更简洁,但题目规定了不准这样做。
\begin{Code}
// LeetCode, Swap Nodes in Pairs
// 时间复杂度O(n),空间复杂度O(1)
class Solution {
public:
    ListNode* swapPairs(ListNode* head) {
        ListNode* p = head;

        while (p && p->next) {
            swap(p->val, p->next->val);
            p = p->next->next;
        }

        return head;
    }
};
\end{Code}

\subsubsection{相关题目}

\begindot
\item Reverse Nodes in k-Group, 见 \S \ref{sec:reverse-nodes-in-k-group}
\myenddot


\subsection{Reverse Nodes in k-Group}
\label{sec:reverse-nodes-in-k-group}


\subsubsection{描述}
Given a linked list, reverse the nodes of a linked list k at a time and return its modified list.

If the number of nodes is not a multiple of $k$ then left-out nodes in the end should remain as it is.

You may not alter the values in the nodes, only nodes itself may be changed.

Only constant memory is allowed.

For example,
Given this linked list: \code{1->2->3->4->5}

For $k = 2$, you should return: \code{2->1->4->3->5}

For $k = 3$, you should return: \code{3->2->1->4->5}


\subsubsection{分析}
无


\subsubsection{递归版}
\begin{Code}
// LeetCode, Reverse Nodes in k-Group
// 递归版,时间复杂度O(n),空间复杂度O(1)
class Solution {
public:
    ListNode *reverseKGroup(ListNode *head, int k) {
        if (head == nullptr || head->next == nullptr || k < 2)
            return head;

        ListNode *next_group = head;
        for (int i = 0; i < k; ++i) {
            if (next_group)
                next_group = next_group->next;
            else
                return head;
        }
        // next_group is the head of next group
        // new_next_group is the new head of next group after reversion
        ListNode *new_next_group = reverseKGroup(next_group, k);
        ListNode *prev = NULL, *cur = head;
        while (cur != next_group) {
            ListNode *next = cur->next;
            cur->next = prev ? prev : new_next_group;
            prev = cur;
            cur = next;
        }
        return prev; // prev will be the new head of this group
    }
};
\end{Code}


\subsubsection{迭代版}
\begin{Code}
// LeetCode, Reverse Nodes in k-Group
// 迭代版,时间复杂度O(n),空间复杂度O(1)
class Solution {
public:
    ListNode *reverseKGroup(ListNode *head, int k) {
        if (head == nullptr || head->next == nullptr || k < 2) return head;
        ListNode dummy(-1);
        dummy.next = head;

        for(ListNode *prev = &dummy, *end = head; end; end = prev->next) {
            for (int i = 1; i < k && end; i++)
                end = end->next;
            if (end  == nullptr) break;  // 不足 k 个

            prev = reverse(prev, prev->next, end);
        }

        return dummy.next;
    }

    // prev 是 first 前一个元素, [begin, end] 闭区间,保证三者都不为 null
    // 返回反转后的倒数第1个元素
    ListNode* reverse(ListNode *prev, ListNode *begin, ListNode *end) {
        ListNode *end_next = end->next;
        for (ListNode *p = begin, *cur = p->next, *next = cur->next;
                cur != end_next;
                p = cur, cur = next, next = next ? next->next : nullptr) {
            cur->next = p;
        }
        begin->next = end_next;
        prev->next = end;
        return begin;
    }
};
\end{Code}


\subsubsection{相关题目}
\begindot
\item Swap Nodes in Pairs, 见 \S \ref{sec:swap-nodes-in-pairs}
\myenddot


\subsection{Copy List with Random Pointer}
\label{sec:copy-list-with-random-pointer}


\subsubsection{描述}
A linked list is given such that each node contains an additional random pointer which could point to any node in the list or null.

Return a deep copy of the list.


\subsubsection{分析}
无


\subsubsection{代码}
\begin{Code}
// LeetCode, Copy List with Random Pointer
// 两遍扫描,时间复杂度O(n),空间复杂度O(1)
class Solution {
public:
    RandomListNode *copyRandomList(RandomListNode *head) {
        for (RandomListNode* cur = head; cur != nullptr; ) {
            RandomListNode* node = new RandomListNode(cur->label);
            node->next = cur->next;
            cur->next = node;
            cur = node->next;
        }

        for (RandomListNode* cur = head; cur != nullptr; ) {
            if (cur->random != NULL)
                cur->next->random = cur->random->next;
            cur = cur->next->next;
        }

        // 分拆两个单链表
        RandomListNode dummy(-1);
        for (RandomListNode* cur = head, *new_cur = &dummy;
                cur != nullptr; ) {
            new_cur->next = cur->next;
            new_cur = new_cur->next;
            cur->next = cur->next->next;
            cur = cur->next;
        }
        return dummy.next;
    }
};
\end{Code}


\subsubsection{相关题目}
\begindot
\item 无
\myenddot


\subsection{Linked List Cycle}
\label{sec:linked-list-cycle}


\subsubsection{描述}
Given a linked list, determine if it has a cycle in it.

Follow up:
Can you solve it without using extra space?


\subsubsection{分析}
最容易想到的方法是,用一个哈希表\fn{unordered_map<int, bool> visited},记录每个元素是否被访问过,一旦出现某个元素被重复访问,说明存在环。空间复杂度$O(n)$,时间复杂度$O(N)$。

最好的方法是时间复杂度$O(n)$,空间复杂度$O(1)$的。设置两个指针,一个快一个慢,快的指针每次走两步,慢的指针每次走一步,如果快指针和慢指针相遇,则说明有环。参考\myurl{ http://leetcode.com/2010/09/detecting-loop-in-singly-linked-list.html}


\subsubsection{代码}
\begin{Code}
//LeetCode, Linked List Cycle
// 时间复杂度O(n),空间复杂度O(1)
class Solution {
public:
    bool hasCycle(ListNode *head) {
        // 设置两个指针,一个快一个慢
        ListNode *slow = head, *fast = head;
        while (fast && fast->next) {
            slow = slow->next;
            fast = fast->next->next;
            if (slow == fast) return true;
        }
        return false;
    }
};
\end{Code}


\subsubsection{相关题目}
\begindot
\item Linked List Cycle II, 见 \S \ref{sec:Linked-List-Cycle-II}
\myenddot


\subsection{Linked List Cycle II}
\label{sec:linked-list-cycle-ii}


\subsubsection{描述}
Given a linked list, return the node where the cycle begins. If there is no cycle, return \fn{null}.

Follow up:
Can you solve it without using extra space?


\subsubsection{分析}
当fast与slow相遇时,slow肯定没有遍历完链表,而fast已经在环内循环了$n$圈($1 \leq n$)。假设slow走了$s$步,则fast走了$2s$步(fast步数还等于$s$加上在环上多转的$n$圈),设环长为$r$,则:
\begin{eqnarray}
2s &=& s + nr \nonumber \\
s &=& nr \nonumber
\end{eqnarray}

设整个链表长$L$,环入口点与相遇点距离为$a$,起点到环入口点的距离为$x$,则
\begin{eqnarray}
x + a &=& nr = (n – 1)r +r = (n-1)r + L - x \nonumber \\
x &=& (n-1)r + (L – x – a) \nonumber
\end{eqnarray}

$L – x – a$为相遇点到环入口点的距离,由此可知,从链表头到环入口点等于$n-1$圈内环+相遇点到环入口点,于是我们可以从\fn{head}开始另设一个指针\fn{slow2},两个慢指针每次前进一步,它俩一定会在环入口点相遇。


\subsubsection{代码}
\begin{Code}
//LeetCode, Linked List Cycle II
// 时间复杂度O(n),空间复杂度O(1)
class Solution {
public:
    ListNode *detectCycle(ListNode *head) {
        ListNode *slow = head, *fast = head;
        while (fast && fast->next) {
            slow = slow->next;
            fast = fast->next->next;
            if (slow == fast) {
                ListNode *slow2 = head;

                while (slow2 != slow) {
                    slow2 = slow2->next;
                    slow = slow->next;
                }
                return slow2;
            }
        }
        return nullptr;
    }
};
\end{Code}


\subsubsection{相关题目}
\begindot
\item Linked List Cycle, 见 \S \ref{sec:Linked-List-Cycle}
\myenddot


\subsection{Reorder List}
\label{sec:reorder-list}


\subsubsection{描述}
Given a singly linked list $L: L_0 \rightarrow L_1 \rightarrow \cdots \rightarrow L_{n-1} \rightarrow L_n$,
reorder it to: $L_0 \rightarrow L_n \rightarrow L_1 \rightarrow L_{n-1} \rightarrow L_2 \rightarrow L_{n-2} \rightarrow \cdots$

You must do this in-place without altering the nodes' values.

For example,
Given \fn{\{1,2,3,4\}}, reorder it to \fn{\{1,4,2,3\}}.


\subsubsection{分析}
题目规定要in-place,也就是说只能使用$O(1)$的空间。

可以找到中间节点,断开,把后半截单链表reverse一下,再合并两个单链表。


\subsubsection{代码}
\begin{Code}
// LeetCode, Reorder List
// 时间复杂度O(n),空间复杂度O(1)
class Solution {
public:
    void reorderList(ListNode *head) {
        if (head == nullptr || head->next == nullptr) return;

        ListNode *slow = head, *fast = head, *prev = nullptr;
        while (fast && fast->next) {
            prev = slow;
            slow = slow->next;
            fast = fast->next->next;
        }
        prev->next = nullptr; // cut at middle

        slow = reverse(slow);

        // merge two lists
        ListNode *curr = head;
        while (curr->next) {
            ListNode *tmp = curr->next;
            curr->next = slow;
            slow = slow->next;
            curr->next->next = tmp;
            curr = tmp;
        }
        curr->next = slow;
    }

    ListNode* reverse(ListNode *head) {
        if (head == nullptr || head->next == nullptr) return head;

        ListNode *prev = head;
        for (ListNode *curr = head->next, *next = curr->next; curr;
            prev = curr, curr = next, next = next ? next->next : nullptr) {
                curr->next = prev;
        }
        head->next = nullptr;
        return prev;
    }
};
\end{Code}


\subsubsection{相关题目}
\begindot
\item 无
\myenddot


\subsection{LRU Cache}
\label{sec:lru-cache}


\subsubsection{描述}
Design and implement a data structure for Least Recently Used (LRU) cache. It should support the following operations: get and set.

\fn{get(key)} - Get the value (will always be positive) of the key if the key exists in the cache, otherwise return -1.

\fn{set(key, value)} - Set or insert the value if the key is not already present. When the cache reached its capacity, it should invalidate the least recently used item before inserting a new item.


\subsubsection{分析}
为了使查找、插入和删除都有较高的性能,我们使用一个双向链表(\fn{std::list})和一个哈希表(\fn{std::unordered_map}),因为:
\begin{itemize}
\item{哈希表保存每个节点的地址,可以基本保证在$O(1)$时间内查找节点}
\item{双向链表插入和删除效率高,单向链表插入和删除时,还要查找节点的前驱节点}
\end{itemize}

具体实现细节:
\begin{itemize}
\item{越靠近链表头部,表示节点上次访问距离现在时间最短,尾部的节点表示最近访问最少}
\item{访问节点时,如果节点存在,把该节点交换到链表头部,同时更新hash表中该节点的地址}
\item{插入节点时,如果cache的size达到了上限capacity,则删除尾部节点,同时要在hash表中删除对应的项;新节点插入链表头部}              
\end{itemize}


\subsubsection{代码}
\begin{Code}
// LeetCode, LRU Cache
// 时间复杂度O(logn),空间复杂度O(n)
class LRUCache{
private:
    struct CacheNode {
        int key;
        int value;
        CacheNode(int k, int v) :key(k), value(v){}
    };
public:
    LRUCache(int capacity) {
        this->capacity = capacity;
    }

    int get(int key) {
        if (cacheMap.find(key) == cacheMap.end()) return -1;
        
        // 把当前访问的节点移到链表头部,并且更新map中该节点的地址
        cacheList.splice(cacheList.begin(), cacheList, cacheMap[key]); 
        cacheMap[key] = cacheList.begin();
        return cacheMap[key]->value;
    }

    void set(int key, int value) {
        if (cacheMap.find(key) == cacheMap.end()) {
            if (cacheList.size() == capacity) { //删除链表尾部节点(最少访问的节点)
                cacheMap.erase(cacheList.back().key);
                cacheList.pop_back();
            }
            // 插入新节点到链表头部, 并且在map中增加该节点
            cacheList.push_front(CacheNode(key, value));
            cacheMap[key] = cacheList.begin();
        } else {
            //更新节点的值,把当前访问的节点移到链表头部,并且更新map中该节点的地址
            cacheMap[key]->value = value;
            cacheList.splice(cacheList.begin(), cacheList, cacheMap[key]);
            cacheMap[key] = cacheList.begin();
        }
    }
private:
    list<CacheNode> cacheList;
    unordered_map<int, list<CacheNode>::iterator> cacheMap;
    int capacity;
};
\end{Code}


\subsubsection{相关题目}
\begindot
\item 无
\myenddot

